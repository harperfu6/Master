\chapter{関連研究}
本章にて,本研究に関連する研究を述べる.本研究では,ユーザが定義する手書きジェスチャを高速に認識し,かつどのような開発環境においても実装可能な軽量なアルゴリズムを開発している.また,本研究において開発したアルゴリズム\$Vはストロークの大きさ,向き,位置に関して識別可能な\$-Family Recognizerである.以上を踏まえ,本研究の関連研究を,一般的な手書きジェスチャ認識アルゴリズムの研究,ユーザ定義に特化した手書きジェスチャ認識アルゴリズムの研究,Progamming by Exampleに関する研究,\$-Family Recognizerに関する研究,手書きジェスチャ認識を可能にするツールキットに関する研究,手書きジェスチャの評価に関する研究に分類する.
本章にて,それらの研究について述べた後,最後に本研究における手書きジェスチャ認識アルゴリズムである\$Vの位置づけについて述べる.

\section{手書きジェスチャ認識アルゴリズム}
文字,ストロークの形状,手書きジェスチャなどの認識は,長く広く研究されている分野であり,多くのアルゴリズムにより実現されてきた,
finite state machines~\cite{Hong00constructingfinite}~(有限オートマトン)は,有限個の状態と遷移と動作の組み合わせからなる論理モデルであり,ある「状態」において,何らかのイベントや条件によって別の状態へ「遷移」することを繰り返すことによって最終的な認識結果を導く.高い認識精度を示すためには,より詳細なモデルの定義が必要となる.
Hidden Markov Models~(HMMs)~\cite{Anderson2004HiddenMM,Sezgin:2005:HES:1040830.1040899, Cao:2005:EOA:1089508.1089540}~(隠れマルコフモデル)は,観測された出力の系列から,内部の状態系列を統計的に推測するためのアルゴリズムである.
neural networks~\cite{Pittman:1991:RHT:108844.108914}は脳機能の特性を計算機上に応用したアルゴリズムであり,大量の学習によってモデルを最適化し,多次元量のデータで線形分離不困難な問題に対して比較的小さい計算量で良好な解を得ることができる.
feature-based statistical classifiers~\cite{Cho:2006:NGR:1711617.1711649,Rubine:1991:SGE:127719.122753}は,大量の学習データによる特徴量をもとに学習データをクラスタリングし,より低次元な認識モデルを生成するためのアルゴリズムである.
ad hoc heuristic recognizers~\cite{Anthony:2010:LMR:1839214.1839258, Wilson:2003:XUI:642611.642706}は,「限定的な」認識アルゴリズムであるため,事前に定義されたジェスチャのみ認識することができる.しかしながら,アプリケーション実行時において,新たな学習データを追加した場合に,新たなヒューリスティック関数を定義しなければならないため,アプリケーションユーザが独自にジェスチャを定義することができない.
template matching~\cite{Kara:2005:ITS:1652319.1652712, Kristensson:2004:SLV:1029632.1029640}は.主に画像処理に利用され,学習データと入力データの画像をそれぞれ走査し,画像上の各位置における類似度を算出するアルゴリズムであり,手書き文字にも応用されている.

これらアルゴリズムはオンライン文字認識及びオフライン文字認識双方においてしばしば用いられるアリゴリズムである.オンライン文字認識とは,ディスプレイなどにペンや指などによって入力された文字を認識する技術の総称であり,オフライン文字認識とは,紙に書かれた文書イメージを光学スキャンし,そのイメージを自動的にコンピュータで処理可能なテキストデータに変換する技術の総称である.しかしながら,これらアルゴリズムは,高い認識精度を示す認識モデルを生成するために膨大な数の学習データが必要であり,素早く手書きジェスチャ入力のテストしたい場合において不向きであるだけでなく,アプリケーションユーザが独自にジェスチャを定義する上で実用的であるとは言えない.また,これらアルゴリズムを実装することは,それぞれの分野に精通していない開発者にとって困難である.

\section{ユーザ定義に特化した手書きジェスチャ認識アルゴリズム}
ユーザ定義の手書きジェスチャを認識できるアルゴリズムのうち,少ない学習データによって高い認識率を示すアルゴリズムを示す.
Rubine classfier~\cite{Rubine:1991:SGE:122718.122753}やDynamic programming~(DTW)~\cite{Tappert:1982:CSR:1664966.1664979}などは,少ない学習データにおいてジェスチャ認識可能なアルゴリズムであるが,Rubine classfierはアルゴリズムに用いられる数式が複雑である上,学習データが2つの場合は認識率が高いとは言えない.また,DTWはアルゴリズムが簡潔であるが,計算量が非常に大きいという問題点がある.計算量を改善したFast DTW~\cite{Salvador:2007:TAD:1367985.1367993}が開発されたが,アルゴリズムは複雑になっており,プロトタイピング環境開発向けとは言い難い.
また,Rubineは多くある特徴量を適切に選ばないと,認識率が低下するという欠点があり,それぞれの特徴量が認識結果に及ぼす影響について深い知識がない場合,特徴量の選定が難しい.このように認識に用いる特徴量を単に増やすことは,それについて識別できることにつながるが,ロバスト性の低下を招くという欠点もある.

\section{Programming by Example}
Programming by Exampleとは,学習データをもとに,認識モデルを自動生成する手法である.
システムに明示的に学習データを例として与え,ユーザの意図や好みを推論することにより,ユーザの意図に沿った処理を行ったり,処理自体を削減したりすることが可能となる~\cite{110003743975}.この手法は手書きジェスチャ認識においても活用されている.
Taranta et al.~\cite{Taranta:2016:RPA:2984511.2984525}は,Gesture Path Stochastic Resampling~(GPSR)という,学習データをもとに人間が書くような手書きジェスチャを生成するためのアルゴリズムを考案した.このように,あるデータをもとにそのデータにより近く,より自然な新たなデータを自動生成するアルゴリズムとして,Syntetic Data Generation~(SDG)~\cite{conf/iccv/NavaratnamFC07,Shotton:2011:RHP:2191740.2192047,Galbally_syntheticgeneration,Lundin:2002:SFD:646280.687684,Gatos:2005:SAK:1106779.1106876,Rodriguez-Serrano:2012:SQH:2240326.2240755,Fischer:2013:GLS:2501115.2501123}があり,これらのアルゴリズムは手書きジェスチャを始めとするパターン認識おいて用いられている.また,Perlin noise~\cite{Perlin:1985:IS:325165.325247}あるいは,SigmaLognormal~\cite{SigmaLognormal}も,手書きジェスチャを自動生成するためのアルゴリズムとして広く利用されている.GPSRはユーザから得られたそれぞれの手書きジェスチャの特徴をもとに,そのジェスチャの認識率が高くなるような自動生成方法を計算することによって,これらのアルゴリズムよりも高い認識率を示した,これらは,学習データを大量に自動生成することができるため,少ない学習データにおいて高い認識率を示す,ユーザ定義に特化した手書きジェスチャ認識アルゴリズムでもある.

\section{\$-Family Recognizer}
\$1~\cite{Wobbrock:2007:GWL:1294211.1294238}は,\$-Family Recognizerの先駆的なアルゴリズムであり,\$1を改良したアルゴリズムは,\$-Family Recognizerと呼ばれている.
\$1は,少ない学習データにより,高い認識率を示すアルゴリズムであるため,ユーザが独自にジェスチャを定義するジェスチャ認識システムを開発することが可能である.また,開発者は,自身のシステムに,簡単な数式のみを含む,およそ100行からなるアルゴリズムを追加するのみによって,単一ストロークからなるジェスチャ認識を行うことが可能であるため,手書きジェスチャ認識をプロトタイピングするような環境において実装可能である.\$1は入力データと学習データの対応する点のユークリッド距離が最小となるような最適な角度を探索することによってジェスチャ認識を行っている.
しかしながら,ストロークを,大きさ,向き,位置に不変にしているため,それらの特徴量に依存するようなジェスチャを認識することができない.したがって,そのようなジェスチャを認識するようなアプリケーションを開発したい開発者は\$1を自身のシステムに採用することができない.
\$1が認識及び識別することができないジェスチャを認識及び識別するために,これまで\$1を拡張した\$-Family Recognizerが開発されてきた.
\$N~\cite{Anthony:2010:LMR:1839214.1839258}は,複数のストロークからなるジェスチャを認識することを可能にし,識別可能なストロークを大幅に増やすことに成功した.\$Nは,ストロークを複数の単一ストロークに分割し,それぞれの単一ストロークを\$1の手法によって認識した.また,自動的に考えられる複数の単一ストロークの組み合わせを計算することによって,ストロークの向きや書く順番にロバストな認識も可能にした.
Quick\$~\cite{Reaver:2011:MQU:2021164.2021183}は,\$1の改良であり,最短距離法によるクラスタリングによって,対応する点のユークリッド距離が最小となる入力データと学習データの組み合わせを効率的に探索し,認識速度を高速化することに成功した.
Protractor~\cite{Li:2010:PFA:1753326.1753654}も\$1に対し,認識速度の面において改良した.最適な角度を探索する際に,Golden Section Search~(GSS)~\cite{Press:1992:NRC:148286}(pp. 397-402)を用いた\$1とは異なり,閉形式解を用いることによって,より高速に探索することを可能とした.
\$N-Protractor~\cite{Anthony:2012:NFA:2305276.2305296}は,\$Nに対し,Protractorの手法を用いることによって,より高速にかつ,より正確に複数のストロークからなるジェスチャを認識することを可能にした
1 cent Recognizer~\cite{Herold:2012:CRF:2331067.2331074}は,\$1よりも高速であり,アルゴリズムも非常に単純であるため実装が容易であるが,認識・識別可能なジェスチャの種類や,認識率の観点から見るとあまり実用的であるとは言えない.
\$P~\cite{Vatavu:2012:GPC:2388676.2388732}は,ストロークを構成する点をPoint Cloudとして扱うことによって,\$N-Protractorよりも,メモリ消費量や認識速度の点において効率的なアルゴリズムを示した.
Penny Pincher~\cite{Taranta:2015:PPB:2788890.2788925}は,ストロークを構成する点間のベクトルを用いることによって,これまでの\$-Family Recognizerと比べて,より高速にかつ,正確に認識することを可能にした.

これらの\$-Family Recognizerは,2次元のストロークからなる手書きジェスチャを認識をすることが可能であり,\$1に対し,アルゴリズムを簡略化したり,認識速度を高速化したり,認識率を高くしたり,認識できるストロークの種類を増やしたりするなどして,繰り返し改善されてきた.

しかしながらこれらのアルゴリズムは,ストロークの特徴量である,大きさ,向き,位置に関して,そのいずれかあるいはすべてについて不変になるようなアルゴリズムを採用することにより,それらの特徴量についてロバストなジェスチャ認識を実現し,その結果認識率や,認識速度の向上を実現してきた.
それらを不変にしないことは,その特徴量について識別できるようになることにつながるが,不変にしない特徴量についてはロバストではなくなるため,認識率の低下や認識速度の低下を招く恐れがある.例えば,1 cent Recognizerはストロークを構成するすべての点の中心座標から,それぞれの点へのユークリッド距離のみを特徴量とし,入力データと学習データの特徴量の差が最小となるストロークを探索しているため,ストロークの形状と書き順は同じであるが,大きさ,向き,位置に関して異なるストロークを識別することは可能である.しかし,それぞれの特徴量についてロバストでないため,それぞれの特徴量について微妙に異なる場合,認識できないことが多々あり,結果的に認識率の低下を招いている.
%認識速度の観点でいえば,DTWは,単純にストロークを構成する点の距離を比較するのみであるため,こちらも,ストロークの形状は同じであるが,大きさ,向き,位置に関して異なるストロークを識別することは可能であるが,認識率を向上させるために,非常に計算量が大きい.

\section{手書きジェスチャ認識を可能にするツールキット}
%プロトタイピング環境向けに開発できるように,
手書きジェスチャ認識を簡単に開発可能なツールキットも開発された~\cite{Henry:1990:IGS:97924.97938,Landay:1993:EEU:259964.260123,Myers:1997:AEN:262050.260628}.特に,SATIN~\cite{Hong:2000:STI:354401.354412}はジェスチャ認識の開発を容易にするだけでなく,ペンベースのユーザインタフェースを採用し,ジェスチャ認識のモデルを手書きによって定義することができるツールキットである.
%Henryら~\cite{Henry:1990:IGS:97924.97938},LandayとMyers~\cite{Landay:1993:EEU:259964.260123},及びAmulet toolit~\cite{Myers:1997:AEN:262050.260628}は,手書きジェスチャを入力として用いたアプリケーションを開発するためのツールキットである.
これらは,開発を手助けするのに非常に強力であるが,対応可能な開発環境が決まっており,自身の環境に適用できない場合がある.

\section{手書きジェスチャの評価研究}
手書きジェスチャは,これまで様々な研究において入力手法として用いられており,曲線などを用いた複雑な構造からなる手書きジェスチャ~\cite{Lu:2011:GAT:1978942.1978972,Li:2010:GST:1866029.1866044,Moran:1997:PIT:263407.263508,Hinckley:2007:ISS:1240624.1240666,Appert:2009:USC:1518701.1519052,Liao:2008:PGC:1314683.1314686,Zeleznik:2008:LCD:1449715.1449741}や,
1本あるいは2本の直線のみからなる単純な構造からなる手書きジェスチャ~\cite{Kurtenbach:1993:LEP:164632.164977}などが用いられている.
Bragdonら~\cite{Bragdon:2011:EAT:1978942.1979000}は,これらの研究において用いられる手書きジェスチャのうち,利用頻度の高い手書きジェスチャを抜粋し,それらをスマートフォンやスマートウォッチに対して入力する際に,どのような場面においてどのような手書きジェスチャが求められているのかを評価した.
Bragdonらは,アイズフリーによる操作及び歩行時における片手操作のみならず,立ち止まって端末を見ながら操作する場面でさえも,曲線などを用いた複雑な構造からなる手書きジェスチャよりも,1本あるいは2本の直線からなる単純な構造からなる手書きジェスチャの方が,認識率が高く,入力するまでの速度が速いため,ユーザによる利用頻度が高いということを示した.

\section{本研究の位置づけ}
本研究における手書きジェスチャ認識アルゴリズム\$Vは,アルゴリズムが簡潔である\$1に,アルゴリズムを少し追加するのみによって実装できるため,パターン認識に関する深い知識がなくとも実装可能である上,どのような開発環境においても実装可能である.また,少ない学習データにより高い認識率を示すため,ユーザ定義手書きジェスチャを認識することが可能であり,既存のユーザ定義に特化した手書きジェスチャ認識アルゴリズムと比較し,認識率及び認識速度において高いパフォーマンスを示すアルゴリズムである.

\$Vは,既存の\$-Family Recognizerとは違い,ジェスチャの形状と書き順が同じであるが,大きさ,向き,位置に関して異なる手書きジェスチャを識別することが可能である上,認識率及び認識速度においてパフォーマンスの低下を最小限に抑えたアルゴリズムである.また,\$Vは,得られた学習データから,ユーザがそれぞれの手書きジェスチャをどのように区別して登録しているのかを推測するProgramming by Exampleの考え方に基づき,大きさ,向き,位置に関して異なる手書きジェスチャを識別するために必要な特徴量に高い重み付けをするという処理を施している.これにより,ロバスト性が維持され,結果的に認識率の低下を抑えている.また,識別するために必要なジェスチャのみに対し,類似度計算を行うことによって,認識速度の低下を抑えている.

これらに加え,\$Vは大きさ,向き,位置に関して異なる手書きジェスチャを識別することが可能であるため,ユーザは,曲線などを用いた複雑な構造からなる手書きジェスチャを多く考える必要がなく,本あるいは2本の直線のみからなる単純な構造からなる手書きジェスチャを用いて,大きさ,向き,位置に関して様々なバリエーションを持たせた手書きジェスチャを考えるだけで良いといった利点がある.





