ペンや指を用いた手書きジェスチャを入力として用いるアプリケーション
手書きジェスチャを認識するための既存アルゴリズムは専門的な知識が必要であったり,認識させるための必要な学習データの数が膨大であったりする.\$1は,アルゴリズムが簡潔である,少ない学習データにおいて高い認識率を示す,認識速度が速い,ロバスト性が高いと行った特徴を持つ手書きジェスチャ認識のためのアルゴリズムであり,様々な開発環境において,アプリケーション開発初学者を含む多くの開発者によって利用されてきた.しかしながら,ユーザ調査により,アプリケーションユーザは,手書きジェスチャの形状や書き順が同じでも,大きさ,向き,位置の違いを利用したジェスチャを入力したいという要望があることがわかった.\$1を始め\$1を改良した多くのアルゴリズムは,認識率や認識速度の低下を危惧し,これらの手書きジェスチャを識別できるように実装されていない.
そこで本論文にて,認識率や認識速度の低下を最小限に抑えつつ,手書きジェスチャの形状や書き順が同じでも,大きさ,向き,位置に関して識別可能なをアルゴリズム\$V示す.\$Vは保管されている学習データの類似度をもとに,類似度計算するジェスチャを選び,認識に用いる特徴量に重み付けすることによって,認識率及び認識速度の低下を抑えた.アプリケーション開発者は,本論文において示されるアルゴリズムを自身のシステムに組み込むことにより,大きさ,向き,位置の違いを利用したジェスチャを入力として用いるようなアプリケーションを開発することができる.
また,\$Vの性能評価のために既存アルゴリズムとの比較実験を行い,その結果から\$Vの有用性を示した.